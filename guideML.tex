\documentclass{amsart}

\usepackage[margin=1.3in]{geometry}
\usepackage[usenames, dvipsnames]{color}
\usepackage{charter}
\usepackage[T1]{fontenc}
\usepackage[charter]{mathdesign}
\usepackage{microtype}

\usepackage{MadLibs}

\hypersetup{colorlinks=true,
			urlcolor=Periwinkle}

\pagestyle{plain}

\title{A Mad Libs Package}
\author{Version 1.0\\Arun Debray\\\today}

\begin{document}
\maketitle

% introduction: where does this document live?

Mad Libs are a(n) \Adjective{} game played by creating a document or story with certain words missing, replaced
with lines guiding the \Noun{} to fill them in. One player \Verb{} the others for words to fill in the blanks, and
then reads them \Adjective{} to other \PluralNoun{} in the room. Often, people play this game at \PluralNoun{}.

For example, \Verb{} the example, ``I got a \Noun{}, baby, and I'll write your name.'' The first player asks the
rest for a noun, and gets ``teapot'' in response; so the final \Noun{} is ``I got a teapot, baby, and I'll write
your name.''

One \Noun{}, I realized that Mad Libs didn't exist in \LaTeX. This \Adjective{} problem needed to be fixed, so I
\Verb{} this package, to make writing Mad Libs stories much more \Adjective{}. Moreover, thanks to the magic of the
\textsf{hyperref} package, it's possible to generate PDFs that can be filled in. Go ahead, write some \PluralNoun{}
in the blanks on this page!
\subsection*{Commands}

Here are the commands you can use.
\begin{itemize}
	\item Simple parts of speech: \verb+\Noun+ creates \Noun, and you can call \verb+\Verb+, \verb+\Adjective+,
	\verb+\Adverb+, \verb+\Preposition+, and \verb+\Interjection+, which have the same effects.
	\item Modified parts of speech: \verb+\EdVerb+ creates \EdVerb, \verb+\IngVerb+ produces \IngVerb, and
	\verb+\PluralNoun+ generates \PluralNoun.
	\item Finally, a few commands aren't exactly parts of speech. \verb+\Number+ produces \Number, and
	\verb+\Exclamation+ creates \Exclamation. Then, \verb+\PersonInRoom+ makes \PersonInRoom, and \verb+\Place+
	creates \Place.
	\item If you want to write your own Mad Libs categories, use \verb+\PartOfSpeech{+\textit{word}\verb+}+. For
	example, \verb+\PartOfSpeech{binky}+ produces \PartOfSpeech{binky}.
\end{itemize}

Questions, comments, or complaints may be directed to \href{mailto:a.debray@gmail.com}{a.debray@gmail.com}.

\end{document}
