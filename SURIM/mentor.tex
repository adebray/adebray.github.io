%\documentclass{amsart}
%\usepackage{geometry,microtype,hyperref}
%\geometry{margin=1in}
%\begin{document}
%\title{The Importance of the Mentor-Mentee Relationship}
%\author{Brian Thomas\\\today}
%\maketitle
Though mentors take many forms (grad students, postdocs, professors, etc.), they all operate in similar ways. They key is that on both sides of the relationship, there are human beings on both ends. Sometimes this gets forgotten. Basic human social interactions are often the reason such relationships succeed or fail.\footnote{Wait, is this a talk on social skills for math majors?}

Consider two types of mentorship (see handout). One extreme is the professor who is very obvious about what sort of direction and timeline the mentee would take, and the other is someone who is much more laid-back. Though most people fall in the middle, the latter type tends to be attracted to academia.

A lot of people prefer this latter type of professor, due to their open-endedness and lack of pressure. Of course, a more structured research project has advantages too; you will probably learn more.

Another important question is despite which is preferable, which would make for more productive research? I personally believe the answer is somewhere in the middle, but there are advantages to the structured approach.

Given that professors are somewhat immutable in this regard, you have to think about this explicitly and know what's going on in order to set a good relationship. Thus, good mentoring relationships can be designed. This is the sort of thing that one must do early and figure out ahead of time --- establish what is expected and what preconceived notions exist.

So, how can this happen? Consider a student who wants to work in a more self-directed manner under a professor who is more directed. This could cause tension about different interests. For example, some students are just interested in experiencing research, but the professor wants to see a paper. If this sort of collision is addressed early, it can be rectified. So, consider: if you had a half-hour meeting on addressing expectations (i.e. you should do this), what topics would be discussed? The final product, the frequency of interaction (which is different for each research project), which direction the research should take (is there a virtual syllabus?), and what stages of polish are expected (what sorts of drafting is expected?). Each of these is very varied amongst different professors, and there are plenty of other differences.

Of course, when conflicts will arise, it is worth knowing how to solve them. How is it possible to start this conversation? One important question is how much time is necessary to sink into a new idea --- often, the reason a professor wants you to work in a specific field is because a different one would require a lot of grunt work beforehand.

So go have this meeting. There will probably be disagreements, but with this half-hour, even as an informal discussion, there won't be a simmering conflict for the end of the summer. Sometimes, people can be intimidated by this --- but remember, what they're doing is not that structured, and outlining expectations is a great, mature thing to bring to a conversation. Remember, they are just as human as you. Do not be nervous. Air expectations now, so they don't cause conflict later. Understand differences and there will be a suitable agreement on frequency of interaction, level of polish, etc.

Notice that while group meetings may be more useful in SURIM, a one-on-one meeting certainly is also a good idea. This helps one get to know a professor, and this sort of thing is harder in a team setting.

If you have questions, drop an email to \texttt{\href{mailto:bthomas@stanford.edu}{bthomas@stanford.edu}}.
%\end{document}
