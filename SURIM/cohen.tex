Suppose $K$ is some imaginary quadratic field with class group $\Cl(K)$ and $\Cl_p(K)$ is the $p$-Sylow subgroup (in this case, the $p^k$-torsion of $K$). The Cohen-Lenstra Heuristcs concern the distribution of $\Cl_p(K)$; specifically, if $A$ is some finite abelian $p$-group, it considers the limit
\begin{equation}
L = \lim_{x\to\infty} \frac{\#\{\text{$K$ is a quadratic field with } \Cl_p(K) \cong A\}}{\#\{\text{$K$ is a quadratic field with } \Disc(K) < x\}}.
\label{uglylimit}
\end{equation}
Let $p$ be an odd prime (when $p = 2$, this is the different subject of genus theory).
\begin{thm}[Cohen-Lenstra Heuristic]
If $p$ is an odd prime and $A$ is a finite abelian $p$-group, then $L$ (in Equation~\ref{uglylimit}) is inversely proportional to $\#\Aut A$: Let
\[(p)_\infty = \sum_{\text{\emph{finite abelian groups }}A} \frac{1}{\#\Aut A} = \prod_{n\ge 1} \frac{1}{1-p^{-n}},\]
so that $L = \frac{1}{(p)_\infty\#\Aut A}$.
\end{thm}
There are several justifications for this seemingly counterintuitive result.

Let $I_1,\dots,I_n$ be a lot of nonzero ideals of $K$.\footnote{To be precise, they are ideals of $\cO_K$, but this abbreviation is commonly used.} They satisfy certain relations of the form $\prod_{j=1}^n I_j^{a_j} = (b)$ (some principal ideal), so $\sum_{j=1}^n a_jI_j = \cO_{\Cl(K)}$ written additively.
\begin{defn}
If $f:G\to H$ is a homomorphism between abelian groups, then the cokernel of $f$ is $\coker f = H/\Im f$.
\end{defn}
If these $a_i$ are chosen randomly and written (over several iterations) in a matrix, then for very large $n$, the cokernel of the matrix is finite.

However, it's not actually possible to pick random integers, so instead random $p$-adic numbers have to be chosen. There are two primary ways to envision the $p$-adic numbers: they can be thought of as a collection of clopen sets as in Section~\ref{skolem}, but also as infinite strings of base-$p$ digits, trailing off to the left. Thus, starting from the units place, one can randomly pick a digit in each place, to generate a random element of $\Z_p$.
\begin{thm}[Friedman-Washington]
\label{randommatrices}
The distribution of cokernels of these random matrices in $\Z_p$ is identical to the Cohen-Lenstra distribution on class groups.
\end{thm}
\begin{thm}[Maples]
A generalization of Theorem~\ref{randommatrices} that claims the distribution does not even have to be random, thugh it must be independent.
\end{thm}
(Though it is more general, this theorem is not terribly useful.)

Another approach to this heuristic is to fix some prime power $p^r$ and randomly generate an abelian group according to the following algorithm:
\begin{enumerate}
\item Randomly generate a multiplication table, and
\item Throw it away if it's not an abelian group.
\end{enumerate}
Obviously this is horribly inefficient and isn't actually implemented on a computer, but has mathmatical significance. To be precise, the probability that a group $B$ randomly generated in this manner is isomorphic to a given $A$ is inversely proportional to $\#\Aut A$.

The automorphism group is sometimes difficult to envision, so here are some examples:

If $A = \Z/n\Z$, then $A$ is generated by a single element, so an automorphism just sends one generator to another, so $\Aut A = (\Z/n\Z)^\times$ and $\#\Aut A = \varphi(n)$.

If $A = (\Z/n\Z)^r$, then choose a basis and send each basis element to something of order $n$: $(a_1,\dots,a_r)$ with $\gcd(a_1,\dots,a_r) = 1$ (which is also equal to the least common multiple of their orders). Consider $n$ of these as column vectors and write them as a matrix $M$. In order to be an automorphism, this matrix must be invertible, so $M\in\GL{r}(\Z/n\Z)$, and thus $\Aut A = \GL{r}(\Z/n\Z)$.

The size of this automorphism group is more complicated, but when $n = p^k$ is a prime power, then $\#\Aut A = \prod_{j=0}^{r-1} \left((p^k)^r - (p^k)^{r-j}\right)$.

For example, $\Aut(\Z/9\Z) = (\Z/9\Z)^\times$, which has size 6, but $\Aut((\Z/3\Z)^2) = \GL{2}\mathbb F_3$, which has size 48. Thus, the heuristic predicts that $\Z/9\Z$ should be about eight times as common as $\Z/3\Z\times\Z/3\Z$ as a class group; however, for small discriminants, this ratio can be overshot.
