Suppose $p = 7$. Then, the squares $\mod 7$ are
\begin{tabular}{c c c|c c c}
$1$&$2$&$3$&$4$&$5$&$6$\\
$+$&$+$&$-$&$+$&$-$&$-$.
\end{tabular}\\
For $p = 23$ it's a bit harder, though still computable without paper:

\begin{tabular}{c c c c c c c c c c c|c c c c c c c c c c c}
$1$&$2$&$3$&$4$&$5$&$6$&$7$&$8$&$9$&$10$&$11$&$12$&$13$&$14$&$15$&$16$&$17$&$18$&$19$&$20$&$21$&$22$\\
$+$&$+$&$+$&$+$&$-$&$+$&$-$&$+$&$+$&$-$&$-$&
$+$&$+$&$-$&$-$&$+$&$-$&$+$&$-$&$-$&$-$&$-$
\end{tabular}\\
And just for one more example, here are the residues $\mod 31$:

\begin{center}
\begin{tabular}
{c c c c c c c c c c c c c c c}
$1$&$2$&$3$&$4$&$5$&$6$&$7$&$8$&$9$&$10$&$11$&$12$&$13$&$14$&$15$\\
$+$&$+$&$-$&$+$&$+$&$-$&$+$&$+$&$+$&$+$&$-$&$-$&$-$&$+$&$-$\\
\hline
$16$&$17$&$18$&$19$&$20$&$21$&$22$&$23$&$24$&$25$&$26$&$27$&$28$&$29$&$30$\\
$+$&$-$&$+$&$+$&$+$&$-$&$-$&$-$&$-$&$-$&$-$&$-$&$+$&$-$&$-$
\end{tabular}\\
\end{center}

Notice that there are more minus signs on the right-hand side of each table than on the left.

Suppose $\chi(n) = \jac{n}{p}$. Then, $\sum_{n=1}^{p-1}\chi(n) = 0$. However, since the residues are distributed unevenly, one obtains
\[
\sum_{n=1}^{\frac{p-1}{2}} \chi(n) =\left\{
\begin{array}{c l r @{\,=\,} l}
1 & p=7 & h(-7)&1\\
3 & p=23 & h(-23)&3\\
3 & p = 31& h(-31)&3.
\end{array}\right.
\]
Well, this looks suspicious. And this suspicion can be confirmed:
\begin{thm}
\label{halfsum}
If $p\equiv 3\mod 8$, then $h(p) = \displaystyle{\sum_{n=1}^{\frac{p-1}{2}}} \chi(n)$.
\end{thm}
In general, the formula is $\sum_{(x,D) = 1}^{D/2}\chi_D(x) = (2-\chi(D))h(-D)$, where $\chi_D(x) = \jac xD$ is called the quadratic character of $D$. Notice that since $\chi(D) = \pm 1$, the conjectured formula is only off by a factor of at most 3.

Now consider
\[
\sum_{n=1}^{p-1}\chi(n)n =\left\{
\begin{array}{r l}
-7 & p = 7\\-69 & p = 23\\ -93 & p = 31\end{array}\right.
\]
Once again, this looks like $\sum_{n=1}^{p-1} \chi(n)n = ph(-p)$. However, if you take only the first half, it evaluates to $\sum_{n=1}^{\frac{p-1}{2}} \chi(n)n = 0$.
\begin{thm}
\label{wholesum}
The more general version of the formula is
\[\frac{1}{|D|}\sum_{(x,D) = 1}^{D-1}\chi(x)x = h(-|D|).\]
\end{thm}
Theorems~\ref{halfsum} and \ref{wholesum} are consequences of a more general class number formula.\footnote{Much of this is going to look like magic and won't be very enlightening without some more background.}
\begin{defn}
Let $K =\Q[\sqrt D]$, $\cO_K$ be its ring of integers, and $D$ be its discriminant. Then, the zeta function $\zeta_{_K}:\C\to\C$ associated with $K$ is 
\[\zeta_{_K}(s) = \sum_{I\subset\cO_K}\Norm(I)^{-s}.\]
\end{defn}
This is where complex analysis enters into number theory. Additionally, the Riemann zeta function is just the function for $K = \Q$: $\zeta_\Z(s) = \sum_{n\in\N} n^{-s}$.
\begin{thm}
\[\lim_{s\to 1} (s-1)\zeta_{_K}(s) = \frac{2\pi}{|\cO_K^\times|}\sqrt{|D|}h(-|D|).\]
\end{thm}
In all but a few small cases, $|\cO_K^\times| = 2$, making this computation relatively straightforward. Interestingly, this theorem was first proven by Dirichlet with quadratic forms, and the modern number-field version came later.
\begin{defn}
The L-function associated with a number field is $L(s,K) = \sum_{n=1}^\infty \chi(n)n^{-s}$, where $\chi$ is the character associated with $K$ (in this case, the Legendre symbol).
\end{defn}
This gives $\zeta_{_K}(s) = \zeta(s)L(s,K)$. Since $\lim_{s\to 1} \zeta(s) = 1$, then the L-function can be used for a class number formula (which is not a closed form, since there is an infinite series involved):
\[L(1,\chi) = \frac{2\pi h(-|D|)}{|\cO_K^\times|\sqrt D}.\]
This is the source of the above formulae.

In the end, there is a class number formula, and Exercise~\ref{formula} has a solution --- albeit not one that can be easily found. It is surprisingly analytic, but also explains a couple observations:
\begin{enumerate}
\item Why the growth rate of the class number formula has to do with the Riemann Hyporthesis, and
\item Since the class number is positive, then there are more quadratic residues in the first half of $\Z/p\Z$ than in the second half. In fact, this \emph{is} the proof; no simpler one has been found.
\end{enumerate}
