\label{dendrogram}
\texttt{ggplot2} is a library in R, so just like any package, you have to install it. Impressively, this can be done on the command line with \texttt{install.packages("ggplot2")}, which asks you to choose a mirror to download from. To load the package, the syntax is \texttt{library(ggplot2)}. The documentation for this is \$50 on Amazon, which is\dots a lot, but on SULAIR, it is available for free for Stanford students. Of course, Google is also helpful.

Some other packages use \texttt{ggplot2} behind the scenes, so it can be helpful to understand their internals.

The dataset for this example is a set of diamond example data. Entering \texttt{str(diamonds)} will list carat, cut, clarity, size, etc. of the 50,000 diamonds in the data. Additionally, setting the seed (i.e. \texttt{set.seed(1410)}) makes the script reproducible, so that it doesn't depend on the time of day or whatever.

Since 54,000 is a bit much to plot, a function called \texttt{sample} allows one to get smaller sections of it to plot. In particular, \texttt{dsmall = diamonds[sample(nrow(diamonds,100),]} returns all of the data about 100 diamonds.

The standard plot function in R is just called \texttt{plot}, and gets the job done, if not terribly prettily. But \texttt{ggplot2} exports a function called \texttt{qplot}, which makes a better graph (e.g. nice background, gridlines, and clearer labels). Additionally, using \texttt{plot} requires one to pass all the variables, as \texttt{plot(diamonds\$carat,diamonds\$price)}. However, \texttt{qplot} is intellignt enough that you can just pass it \texttt{qplot(carat,price,data=diamonds)}. It also supports lograithmic plots as the last argument \texttt{log='x'} (or \texttt{xy}, or whatever).

%This would be pretty nice with graphics...
Unsurprisingly, there are also arguments to change the color of the plotted dots, their shape, and the labels on the $x$ and $y$-axes. This makes use of a function called \texttt{I} in R which allows R to use the argument as-is. For example, the shape is set as \texttt{shape=I(18)} (indicating that 18 is not a variable or whatever) and the color as \texttt{color=I("blue")}. One can color by a factor or some other variable (e.g. \texttt{color=price}), or do the same thing with shape. One can set a vatriable called \texttt{alpha}, which is a fraction corresponding to transparency (and lower \texttt{alpha} implies lower opacity).

These plots can be saved --- it probably won't be as necessary if you integrate plots in directly with Sweave and \texttt{knitr}, but it might yet be useful. The \texttt{setwd} function changes where the file is saved, and the \texttt{png} function saves it as a png. RStudio makes this even easier, with a one-click button to export a plot as an image.

Now let's do regressions! The \texttt{geom} parameter in the arguments of \texttt{qplot} allows one to set regressions. One can also set the confidence level. The implementation takes the form of a local polynomial regression (loess). The degree of ``localness'' is controlled by a variable called \texttt{span}. If there are more than 1000 points, this can be fairly computer-intensive, and so a different model is used. However, the type of regression can be set (eg. \texttt{lm} for a linear regression, \texttt{rlm} for a linear regression less influenced by outliers, etc.).

\texttt{qplot} can do some boxplots (i.e. box-and-whisker plots) for various quantities. Once again this involves the \texttt{geom} argument. Similarly, one can make a histogram with \texttt{geom="histogram"}. Since this is very dependent on the width of the bars, one can set a variable called \texttt{binwidth} to adjust the sensitivity of the histogram. Histograms have various arguments that control color, but it's generally simpler to say \texttt{facets=color}, which graphs each set of diamonds of a given color separately, and is typically easier to read. One can also generate various scatter plots with \texttt{pairs}. WIth the right arguments, you can set each piece of data against the next, which makes it really easy to identify relationships among the various pieces of data. Bar charts are rare in statistics, but they can be handled by \texttt{qplot} as well. Similarly, path and line plots are possible, and \texttt{qplot} automatically sorts this sort of data. Path plots can also be used to project 3-dimensional data on a 2-dimensional plane.

Another useful application of this is heirarchical clustering. This is a metric on a set of data that is like cladistics in biology, as every point is ``next to'' either another point or a set of points within a certain distance. The diagram that demonstrates this is called a dendrogram.

The function in question is called \texttt{hclust}, which when plotted makes a nice diagram indicating which data points are more related. Of course, it can be rather non-pretty if there's too much data, but interesting relations can nontheless be identified. Dendrograms can be done for data that wouldn't necessarily make sense to plot, but it can still be useful in idenfifying connections between things.

Using the function \texttt{cut(as.dendrogram,...)} allows for some higher-level clustering (e.g. graphing branches rather than leaves to make the diagram less busy). There are methods to choose the best representatives of a branch, but that is beyond the scome of the lecture here.

One can also cluster two variables at the same time, which is a function called \texttt{heatmap}, which graphs two dendrograms on $x$ and $y$-axes. The color means intensity (white imples larger, rather like in an actual heat map). The formal term for this sort of term is biclustering. Cutting the dendrograms here works just as well, and makes the heat map much more readable.
