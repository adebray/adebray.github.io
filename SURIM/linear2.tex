A list of topics I was interested in pursuing within linear recurrences:
\begin{enumerate}
\item How fast can a squence (or its absolute value) grow? This would be solved by studying its closed form.
\item How can sequences be decomposed or combined into other sequences?
\item What is the structure of the zero set? For a given set $S$, is there a recurrence that satisfies $a_n = 0$ when $n\in S$? Thanks to the Skolem-Mahler-Lech theorem below, there is less to do here.
%I need to learn how to hyperlink.
\item Do the difference tables for a recurrence relation converge? I would guess so, but I haven't proven it.
\item What sorts of decimals can a recurrence relation represent? (Some work done, see below)
\item Is there a function whose series of derivatives at a point matches a recurrence relation? What can be said about the function? This might relate to differential equations.
\item How does one determine whether a sequence is bounded or unbounded? Can the characteristic polynomial be helpful here rather than going all the way to closed forms?
\end{enumerate}
\begin{ex}
Show that any polynomial sequence is a linear recurrence sequence, or that (more generally) the sum or product of two recurrence sequences is also a linear recurrence sequence. Hint: This is related to the closure of $\A$ (i.e. the roots in $\mathbb{C}$ of $x^2+bx+c$, $b,c\in\Z$. $\mathbb{Q}\cap\A = \Z$).
\end{ex}
\begin{ex}
What is the minimum degree of a recurrence relation with zeros given by some zero set?
\end{ex}
\begin{ex}
Are linear recurrences $\mod n$ periodic? What is a sharp\footnote{i.e. there exists such a sequence with that bound as a period.} bound on the period of such a sequence? Hint: Look at for which degrees/moduli a given bound is possible or impossible. Even $\mod 2$ has some interesting things to share.
\begin{proof}[Partial Solution]
Suppose a linear recurrence is of degree $k$ and we are interested in $\mod{} n$. Then, there are only $n^k$ possible inputs into the recurrence, since the recurrence can be thought of as a function of $k$-tuples to $k$-tuples. Thus, the function must repeat itself, since there are only a finite number of possible inputs. The recurrence must be periodic, then, with period at most $T = kn^k$.
\end{proof}
\end{ex}
