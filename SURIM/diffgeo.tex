%WHere are your pictures?
Consider some manifold $M^n\subset \R^N$ with some metric $g$. Consider the tube of radius $r$ around this manifold:
\[T(r) :=\{x\in\R^N:g(x,M) = r\},\]
which is a manifold of codimension 1. This tube is in a sense the product $M \times S^{N-n-1}$, though if $M$ is not completely straight this is incorrect. However, it is roughly so, since the volume $V(r) = \vol(T(r)) \sim r^{N-n-1}$.

More explicitly, Weyl's Tube Formula gives the volume as polynomial in $r$: $V(r) = \sum_{j=0}^{N-n-1} a_jr^j$, where the $a_j$ are geometric invariants. $a_0 = \vol(M)$, which is a geometric qualtity, but $a_{N-n-1} = \chi(M)$, the manifold's Euler characteristic, which is topological.

In $\R^3$, this simplifies to $V(r) = \chi(M)r^2 + \Area(M)$ for a surface and $V(r) = 2\pi r\ell(M)$ for curves. This latter case is interesting because it looks exactly like if the tube were the direct product $M\times S^{N-n-1}$.

This model has some interesting and unexpected applications in statistics.

Differential geometry also uses hyperbolic space. One can think of hyperbolic space $\H^n\subset \R^n$ as an open disc in which lines in $\H^n$ are arcs of circles in $\R^n$ and parallel lines are defined as those that do not intersect. Thus, one line can have many parallels, some of which intersect each other.

The Riemannian metric for hyperbolic space is
\[g = \frac{\sum_{i=1}^n \ud x_i^2}{\left(1- \sum_{i=1}^n x_i^2\right)^2},\]
which results in a constant curvature of $-1$. it also leads to a Laplacian, which is different from $\D g_{_{\R^n}} = \sum_{j=1}^n \frac{\partial^2}{\partial x_j^2}$. However, it enables one to think of bounded harmonic functions in $\H^n$.

Liouville's Theorem states that any bounded harmonic function in $\R^n$ is constant, but hyperbolic space (which is topologically just an $n$-dimensional space) looks very different; there are plenty of nonconstant harmonic functions, and any continuous function on ``the sphere at infinity'' is the limit of some harmonic function. This is indicative of the different geometries of $\R^n$ and $\H^n$, which is also represented by the fact that the volume of the unit ball in $\H^n$ grows exponentially with the radius, rather than as a polynomial.

Instead of considering strictly harmonic functions, however, one can consider solutions to the equation $(\D + \l^2)u = 0$ in $\R^n$, which have applications to scattering theory in mathematical physics. Looking from infinity, $u$ will satisfy the Helmholtz equation
\[
u \sim r^{\frac{1-n}{2}} e^{i\l r}a(\theta) +r^{\frac{1-n}{2}}e^{-i\l r}a(-\theta)
\]
where $\theta\in S^{n-1}$ and $a$ is some function. Physically, this means sending waves through Euclidean space doesn't change them; they come out like they went in. However, if you add some obstacle $K$, an additional requirement of $u|_{\partial K} = 0$ as well. This is similar, but the equation then becomes
\[
u \sim r^{\frac{1-n}{2}} e^{i\l r}a(\theta) +r^{\frac{1-n}{2}}e^{-i\l r}b(\theta)
\]
for some other function $b$. Physically, this means the waves will get scattered by the obstacle $K$, and this equation can be backed up by physical data. Given scattering data, what can one determine about $K$? This problem has applications in seismology, oil detection, etc. Additionally, one can study the scattering operator for this system, given by $b = S(\l) a$.

This has applications in, of all places, number theory. Consider the modular surface $\H^2/\SL{}(2,\Z)$. This looks like a once-punctured torus with a cusp at infinity, and has constant negative curvature. This surface has a hyperbolic Laplacian associated with it, and let $S(\mu)$ be the scattering operator for the equation
\[\left(\D g +\left(\frac{1}{4} +\mu^2\right)\right)u = 0.\]
Looking from the cusp at infinity and sending a wave in, it will reemerge with some function that determines $S$. Considering $\mu\in\C$, $S$ will have poles, whose location is determined by the Riemann Hypothesis.
