\label{present}
%More importantly, some backwards compatibility is compromised...
Obviously it's very important to be able to explain things to people, both within and without of your field. For just one example, at conferences, one has a short time to explain one's topic of research to some people with short attention spans. Additionally, it should be clear that with practice and effort, one can become much better at presentations.

Here are some general pointers to help at a presentation:
\begin{itemize}
\item Even in a room full of people who are familiar with your topic, tell them what you are going to talk about! Even if your main subject is the last part of the talk, it is still important to ensure that nobody is trying to figure it out during the talk. If they know what to expect, they are more likely to follow along.
\item Relatedly, a good speech can begin with ``I am going to tell you this,'' then talk about exactly what it claimed to, and then end with ``I have told you this.'' In this scheme, the last idea should also be particularly interesting.
\item Given that you're not Mozart, your mind does not work in a linear manner. If a certain thought process led to a given talk, which aspects of that process do you want to include? Of course, not everything belongs, and selecting these may be the hardest part of preparing a talk. What is the most interesting to you? What is the most interesting to others? And what order is it said in?
\item Don't talk to the blackboard, especially if you would do so softly. Looking at the audience will make the lecture or class much more interesting; people will see that you care. Similarly, don't read from a paper while avoiding eye contact with the audience; it has the same effect.
\item Given a limited amount of time, it is extremely important that you talk about a small number of things. This is related to setting up correctly, but it also helps people understand.
\item Practice talks. Just as when one reviews a paper, this helps know the timing and speed of the talk. It can also tell you if you did something incredibly right. Though this sounds abstract, having feedback can help inspire confidence and such.
\item Nervousness is perfectly normal; one cannot fix it, but should instead just learn to deal with it. However, rocking back and forth, playing with hair, or other tics distracts or concerns people and reduces the effect of the talk. Practicing helps with this immensely. (Other examples of tics include saying ``like'' too frequently or fiddling with one's pen or glasses.)
\item It is important to behave professionally. In particular, if given a certain amount of time, do not go over. (Of course, people routinely ignore this.) This is particularly true if others are talking after you. Additionally, people like if you finish early.
\item Don't forget questions and answers. If you don't know the answer, admit it --- if you don't, they will know. As a consequence, you should be prepared for questions, of course. And be sure to avoid the questions that turn into talks themselves; know how to interrupt so you can talk afterwards. Field questions during and after the talk, and build them into the time.
\item If not talking, listen and keep eye contact in order to help the presenter. Ask questions if you have good ones. It is acceptable to have a planted question, apparently.
\item Have all the material with you, even if only some of it wil be presented. This should be obvious (answering questions, further discussion, etc.).
\item Good posture goes a long way, with lower shoulders. Focusing on the exhale of a given breath can help limit nervousness.
\item Try not to cancel talks except in emergencies. Sometimes tiredness helps one give a great talk!
\end{itemize}
