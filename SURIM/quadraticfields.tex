As a quick refresher, a ring is a set that allows for some operations of addition, subtraction, and multiplication, and a field also allows for division.

\label{qfields}
\begin{defn}
A quadratic field is a set $\Q[\sqrt{d}] = \{a = b\sqrt{d}:a,b\in\Q\}$, where $d\in\Z$ is not a square. Generally, one can assume $d$ is square-free as well.
\end{defn}%Q adjoin D

Inside $\Q[\sqrt{d}]$ there are a bunch of subrings, given by choosing 2 elements $\alpha,\beta\in\Q[\sqrt{d}]$ such that $\Z\alpha+\Z\beta$ form a ring. Not all choices of $\alpha$ and $\beta$ will work, but in particular, if $\alpha = 1$, then this works iff $\beta$ is integral.

\begin{defn}
An element in $\Q[\sqrt{d}]$ is integral if it is the root of a monic polynomial with integer coefficients.
\end{defn}
For example, in $\Q[i]$, $-1+i$ is integral, since it is the root of $x^2-2x+2$, but $i/3$ is not.
\begin{defn}
If $\beta$ is integral, the ring $\Z+\Z\beta$ is called an order.
\end{defn}

The largest possible order in $\Q[\sqrt{d}]$ is simply the one containing all of its integral elements. Assuming $d$ is square-free, then this order, called the ring of integers,\footnote{``One ring to rule them all, one ring to bind them\dots''} is written $\cO_{\Q[\sqrt{d}]}$ and given by\footnote{The shorthand $(p)$ will be used for $\mod p$.}
\[
\cO_{\Q[\sqrt{d}]} = \left\{
\begin{array}{l l}
\Z[\sqrt{d}] & \text{ if } d\equiv 2,3\ (4)\\
\Z\left[\frac{1+\sqrt{d}}{2}\right] & \text{ if } d\equiv 1\ (4).
\end{array}
\right.%}
\]
\begin{defn}
The discriminant of a field is 
\[
\Disc\left(\Q\left[\sqrt{d}\right]\right) = \left\{
\begin{array}{l l}
d & \text{ if } d\equiv 1\ (4)\\
4d& \text{ if } d\equiv 2,3\ (4).
\end{array}\right.%}
\]
(Assuming $d$ is square-free, it cannot be $0\mod 4$.) In particular, the discriminant is always $0,1\mod 4$. Other discriminants do correspond to quadratic fields, but they don't behave as nicely.
\end{defn}
\begin{defn}
The trace of some $\alpha\in\Q[\sqrt{d}]$ is $\alpha +\bar\alpha$ (where the conjugate of $a + b\sqrt{d}$ is $a - b\sqrt{d}$). Thus, if $\alpha = a + b\sqrt{d}$, then $\Tr \alpha = 2a$.
\end{defn}
\begin{defn}
The norm of $\alpha\in\Q[\sqrt{d}]$ is $\alpha\bar\alpha$, and if $\alpha = a+b\sqrt d$ as before, then this is $a^2-db^2$. Interestingly, this means $\Norm(3) = 3^2$.
\end{defn}
A given order has a canonical choice for $\beta$: if $R$ is an order in $\Q[\sqrt{-d}]$, where $d > 0$, then $R = \Z + \Z n\sqrt{-d}$ or $R = \Z + \Z n \left(\frac{1+\sqrt{-d}}{2}\right)$ (depending on modulus) for a unique $n\in\N$. Then, $\Disc(R) = n^2\Disc(\Q[\sqrt{-d}])$.

\begin{defn}
In a ring $R$, an element $s$ is a unit if some $t\in R$ satisfies $st = ts = 1$.
\end{defn}
In $\Z$, the only units are $\pm 1$, for example.

In $\Z$, there is unique factorization up to units and reordering (since $22 = (2)(11) = (-2)(-11)$). This is possible in many rings, including also $\Z[i]$, $\Z[\sqrt{-2}]$, and $\Z\left[\frac{1+\sqrt{-163}}{2}\right]$, but not everywhere. In particular, working in $\Z[\sqrt{-5}]$, one has $6 = (2)(3) = (1+\sqrt{-5})(1-\sqrt{-5})$.
\begin{defn}
If $R$ is some ring and $p\in R$ is nonzero and not a unit, then $p$ is prime if whenever $p\mid rs$, then either $p\mid r$ or $p\mid s$.
\end{defn}
However, in this case with $\Z[\sqrt{-5}]$, each of these four factors is prime! Unique factorization is nice, but doesn't hold true in general.

The way to get around this issue is to consider ideals, which are sets of numbers, rather than just looking at numbers.
\begin{defn}
A subset $I$ of some commutative\footnote{If the ring is noncommutative, things get slightly more interesting. One has left ideals, for which the absorption property states that if $a\in I$, $r\in R$, then $ar\in I$, right ideals, for which it requires that $ra\in I$, and ideals, which are sets that are both left and right ideals.} ring is an ideal if it an abelian group under addition and it obeys the absorption property, that if $a\in I$, $r\in R$, then $ar\in I$.
\end{defn}
For example, the ideals in $\Z$ are $I = n\Z$ for $n\in\Z$. For shorthand, this can be written $(n)$, indicating that the ideal is generated by $n$. Then, for ideals generated by more than one element, $(a_1,a_2) = Ra_1 + Ra_2$ and so on.
\begin{defn}
An ideal $P$ in some ring $R$ is prime if $P \subsetneq R$ and whenever $r,s\in R$ and $rs\in P$, then either $r\in P$ or $s\in P$.
\end{defn}
In $\Z$, the ideals are $(p)$ for prime $p$ and $0 = (0)$, or the zero ideal. However, in $\Z[\sqrt{-5}]$, there are more interesting examples, such as $I = (2,1+\sqrt{-5})$ --- in particular, not all ideals can be generated by a single element.
\begin{defn}
A principal ideal is one that can be generated by a single element.
\end{defn}
\begin{ex}
Prove that $(2,1+\sqrt{-5})$ is not a principal ideal.
\end{ex}
\begin{proof}[Solution:]
Suppose that $I = (2,1+\sqrt{-5})$ is a principal ideal. Then, $I = (\alpha)$ for some $\alpha = a + b\sqrt{-5}$, $2 = m\alpha$, and $1+\sqrt{-5} = n\alpha$ for $k_1,k_2\in\Q[\sqrt{-5}]$ given by $m = m_1 + m_2\sqrt{-5}$ and $n = n_1 +n_2\sqrt{-5}$, with $a,b,m_1,m_2,n_1,n_2\in\Q$. Thus,
\begin{align*}
2 &= m_1a + m_2a\sqrt{-5} + m_1b\sqrt{-5} -5m_2b\\
1+\sqrt{-5} &= n_1a + n_2a\sqrt{-5} +n_1b\sqrt{-5} -5n_2b.
\end{align*}
This creates a system of equations given by
\begin{alignat*}{2}
m_1a-5m_2b &= 2\qquad &m_2a +m_1b &=0\\
n_1a - 5n_2b &= 1 &n_2a +n_1b &=1
\end{alignat*}
over $\Q$. Since this has no solutions, then $I$ is not a principal ideal.
\end{proof}
Let $I_K$ be the set of nonzero ideals of some field $K$ (which is in this case $\Q[\sqrt{-d}]$). An equivalence relation can be given by $I \sim J$ if there exist nonzero elements $\alpha,\beta\in\cO_K$ such that $\alpha I = \beta J$. In $\Z$, all nonzero ideals are equivalent, since $n(m) = m(n)$ when $m,n\in\Z\setminus 0$.
\begin{defn}
A principal ideal domain is a ring in which every ideal is generated by a single element. In rings of integers, this is equivalent to the statement that all nonzero ideals are equivalent.
\end{defn}
The following is a major theorem of algebraic number theory:
\begin{thm}
If $C_K$ is the set of equivalence classes of ideals of $\cO_K$, then it is finite for any $K$ and is a group under multiplication given by $IJ$ is the smallest ideal such that if $r\in I$ and $s\in J$, then $rs\in IJ$.
\end{thm}
This becomes relevant because there exists a bijection that preserves group structure between $C_K$ and the equivalence classes of positive definite quadratic forms. Since it's so much easier to see the group structure from this perspective, this is the modern language used for this stuff.

From linear algebra, one can conclude that every ideal in the lattice $\Z + \Z\delta$ can be generated by two elements, so pick two generators of some ideal $I = (\alpha,\beta)$ and send it to $\Norm (\alpha x + \beta y)$. This form will not be primitive, but then it is possible to divide by the gcd to make it so. In the other direction, the quadratic form $ax^2+bxy+cy^2$ is sent to the ideal $\left(a,\frac{b-\sqrt{d}}{2}\right)$, where $d$ is the discriminant.

An alternate bijection is given by rewriting $(\alpha,\beta)$. Swap them if necessary so that $\Im\left(\frac{\alpha}{\beta}\right) >0$. Then, apply the bijection to $\left(\frac{\alpha}{\beta},1\right)$. However, if you attempt to apply either of these to the principal ideal, strange things can happen since the quadratic field is a vector space and its ring of integers is a module. Changing $\alpha$ and $\beta$ may affect this. Different choices can generate the same module, but different ideals. This affects no ideals which have two generators.

To give some sense behind the bijection,
\begin{align*}
\Norm\left(ax,\frac{b-\sqrt{d}}{2} y\right) &= \Norm\left(ax + \frac{b}{2}y -\frac{\sqrt{d}}{2}y\right)\\
&= \left(ax+\frac{b}{2}y - \frac{\sqrt{d}}{2}y\right)\left(ax+\frac{b}{2}y + \frac{\sqrt{d}}{2}y\right)\\
&= \left(ax+\frac{b}{2}y\right)^2 - \left(\frac{\sqrt{d}}{2}y\right)^2\\
& = a^2x^2 + abxy + \frac{b^2}{4}y^2 -\frac{d}{4}y^2\\
&= a(ax^2+bxy+cy^2).
\end{align*}
Though unique factorization does not always work on rings, unique factorization of ideals into prime ideals does. For example, consider $(5)\subset\Z[i]$. Since $5 = 1^2+2^2 = (1-i)(1-2i)$, then $(5) = (1-i)(1-2i)$. This algorithm holds in general if a number can be written as the sum of two squares (i.e. $\jac{-1}{p} = 1$; see Section~\ref{use}). However, proving the uniqueness of this factorization is much trickier.
\begin{ex}
$(5) = (2+i)(2-i)$ as well, so these prime ideals must be equivalent to $(1-i)$ and $(1-2i)$. Which corresponds to which?
\end{ex}
\begin{proof}[Solution:]
$\frac{2+i}{1-2i} = i,$ so $(2-i)=(1-i)$ and $(2+i) = (1-2i)$.
\end{proof}
On the other hand, $(7)$ is a prime ideal in $\Z[i]$, since it can't be written as a sum of squares.

If $p$ is an odd prime, there are three possibilities for the factorization of $(p)$. $(p)$ can be prime (so the ideal is inert), it can be decomposed as $(p) = \fP_1\fP_2$ into a product of two primes (called split), or it can be written as $(p) = \fP_1^2$ for some other prime $\fP_1$, in which case it is called ramified.\footnote{The symbol $\fP$ isn't encountered terribly often, so for any \LaTeX{} users following along, it can be generated with the \texttt{mathfrak} math fonts as an uppercase P.} In particular, for quadratic fields with discriminant $d$, one has $(p)$ inert if $\jac{d}{p} = -1$, $(p)$ split if $\jac{d}{p} = 1$,\footnote{i.e. $p$ is representable by some quadratic form with discriminant $d$ --- another interesting connection.} and $(p)$ ramified if $\jac{d}{p} = 0$.
\begin{ex}
Given a representation of $p$ by some quadratic form $\Q[\sqrt{-d}]$, how can this be used to obtain a factorization of $(p)$?
\end{ex}
